\documentclass[class=scrbook, crop=false]{standalone}
\usepackage[subpreambles=true]{standalone}
\ifstandalone
    \input{../settings+/settings}
\fi

% ----------------------------------------------------------------------------
%                               Summary
% ----------------------------------------------------------------------------
\begin{document}

\chapter{Summary}
\label{Chapter::Summary}
    In this chapter, there is a short summary? results of the thesis are presented and evaluated as well as, the approach of this thesis.

In the experiments random forest performed remarkably well. 
When looking at MAE it outperformed all the other models. 
This might be due to the additional information passed to random forest. 
Only random forest and ARIMA received information about past days and past weeks for the same timestamp. 
With Load and Generation behaving periodically with a period of one day and one week, this information might have a large impact.

The ARIMA model has access to the same information but the model does not seem to be able to capture the complexity of the time series.
Another reason why random forest outperformed the other models is because the hyperparameters were tuned a little. 
For this method the best model was determined by using random search, testing 20 different configurations in the process.
The hyperparameter tuning used tried to reduce mean squared error.

This makes it more impressing that xLSTM has a lower RMSE than random forest, even without hyper parameter tuning.

Most of the models were not able to beat the naive intraday index ID1 when looking at MAE. 
This might be due to the fact that information about current prices are lacking. 
The ID1 is composed of market decisions done by BRPs, which have access to market prices.
Additionally no information about balancing reserves was included in the training process for the machine learning models.

The performance of the more complex time series predictors can also be explained by the lack of hyperparameter tuning. 


% Show your results and analyze them
% Explain why the results are the way they are and not different
% Go into possible measurement errors and why they occurred
% Look for anomalies and explain why they occur
\section{Analysis or Evaluation of Something}
\label{Section::Short Analysis or Evaluation of Something}    
    Text

\end{document}

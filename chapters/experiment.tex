\documentclass[class=scrbook, crop=false]{standalone}
\usepackage[subpreambles=true]{standalone}
\ifstandalone
    \input{../settings+/settings}
\fi

% ----------------------------------------------------------------------------
%                               Results
% ----------------------------------------------------------------------------
\begin{document}

\chapter{Experimental Setup} % Overview text
\label{Chapter::Experimental_Setup}
   This chapter contains an explanation of how the experimental setup was done, what parameters had to be controlled, and the equipment used.

% Explain the setup of your experiment in such a way that others could repeat it
% First, explain the environment of the experiment
% Second, explain the tools used
% Lastly, how is everything setup for the experiment, what are the variables?
% Do not discuss results here !!!
\section{Experiment: Out of the box performance}
\label{Section::Experiment_out_of_the_box}

In this experiment different Time Series predictors are compared against each other.
The models used for this are naive, random forest, arima, iTransformer and xLSTM.
The naive model is a benchmark that predicts the imbalance price to be the intraday index ID1. 

For random forest and arima some additional lagged features are created to give information about the data being time series data.
For each feature lagged features are introduced, the features are lagged by 15 minutes, 30 minutes, 45 minutes and an hour.
Also weekly lagged features are used. The features are lagged by one and by two weeks.

For the iTransformer model and xLSTM model the hyperparameter configuration can be found in table \ref{Table::raw_hyperparameters}

\begin{table}[]
\centering
\begin{tabular}{l|l|l}
 Hyperparameter name & iTransformer & xLSTM   \\\hline
 Learning rate & 10^{-3} & 10^{-3}  \\
 Sequence length & 96 & 96  \\
 Batch size & 16 & 32  \\
 Embedding dimension & 512 & 128  \\
 
\end{tabular}
\caption{Hyperparameter configuration for iTransformer and xLSTM}
\label{Table::raw_hyperparameters}
\end{table}

\section{Experiment: xLSTM different xLSTM block structures}

In this experiment different xLSTM configurations were compared. 
Three different xLSTM models were trained and their results compared.
Each model contains an xLSTM block with 48 modules.
The different models are named xlstm\_n\_m, where n:m denotes the fraction of mLSTM to sLSTM blocks.
The configurations tested were xlstm\_1\_0, xlstm\_7\_1 and xlstm\_0\_1. 

\section{Experiment: iTransformer variable sequence length}
   
   In this experiment iTransformer models were trained to inspect how sequence length changes model performance.
   For this the sequence lengths 96, 192 and 384 were used. Other than that the hyperparameters were the same as shown in table \ref{Table::raw_hyperparameters}   
   
\section{Experiment: iTransformer multiple target variables}

  In this experiment the iTransformer model was trained on multiple target variables.
  During the training the loss is calculated on all of the target variables. 
  During the testing only the prediction for imbalance price is used to calculate the error metrics.
  A list with the target variables for each configuration can be found in table \ref{Table::Mutliple_targets}, the other hyperparameters are not changed from \ref{Table::raw_hyperparameters}.
  
  The configuration \textit{intraday indices} is using the intraday indices as a bonus target variables. 
  The configuration \textit{entsoe} uses the actual generation of not renewable energy sources as bonus target variables. 
  These include \textit{fossil gas, fossil hard coal} and \textit{lignite}.
  
  \begin{table}[]
\centering
\begin{tabular}{l|l|l}
 Configuration name & iTransformer & xLSTM   \\\hline
 Intraday indices& 10^{-3} & 10^{-3}  \\
 Entsoe data & 96 & 96  \\
 Netzregelverbund & 16 & 32  \\
 Embedding dimension & 512 & 128  \\
 
\end{tabular}
\caption{Hyperparameter configuration for iTransformer and xLSTM}
\label{Table::raw_hyperparameters}
\end{table}

\section{Experiment: iTransformer longer forecast horizon}

\section{Experiment: iTransfomer combination}



\end{document}

\documentclass[class=scrbook, crop=false]{standalone}
\usepackage[subpreambles=true]{standalone}
\ifstandalone
    \input{../settings+/settings}
\fi

% ----------------------------------------------------------------------------
%                               Outlook
% ----------------------------------------------------------------------------
\begin{document}

% What could be improved?
\section{Outlook}
\label{Section::Outlook}

Access to more current price data would likely improve prediction quality.
The intraday indices are already one of the most influential features for random forest even tough the data is delayed by one day.
Having information about current prices or bids would provide additional information. 
With prices and bids the development of the price for a certain quarter hour could be tracked.
Capturing possible trends in price development might boost prediction quality, especially in quarter hours where the imbalance price is extreme.

Another promising data source that will probably improve the prediction is information about balancing reserves.
Due to the fact that the balancing energy market also is an auction, the balancing energy price is not always the same.
For each quarter hour the cost for obtaining balancing energy can differ.
One way of tackling this problem is to only predict the imbalance volume, not the imbalance price.
Another option is a two step approach, first prediciting the imbalance volume and then prediciting or calculating the imbalance price based on the reserve capacity auction results.
A third option would be to include the information about reserve auction bids into the models. 

Another approach for increasing prediction quality would be to include periodic data into the dataset.
The load and generation profiles show periodic tendencies along different axes. 
One example is the solar generation which is 0 in the night and high in the early afternoon. 
There is also a weekly periodic pattern. 
The weekends show different load profiles than weekdays.
This information can be included into the model by artificially concatenating these sequences into one sequence or by including new features which contain the shifted information.
\end{document}

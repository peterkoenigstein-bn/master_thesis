\documentclass[class=scrbook, crop=false]{standalone}
\usepackage[subpreambles=true]{standalone}
\ifstandalone
    % WARNING: Proceed with caution!

% -----------------------------------------------------------------------------------
% For package standalone
% -----------------------------------------------------------------------------------
\usepackage{import}

% -----------------------------------------------------------------------------------
% Language and typeset
% -----------------------------------------------------------------------------------
\usepackage[ngerman, english]{babel}

\usepackage{subcaption}
% Umlauts and other special characters (UTF-8)
% \usepackage[utf8]{inputenc}
\usepackage{fontspec}
\setsansfont{Arial}
% \usepackage[T1]{fontenc}  % Enable accented characters and umlauts
% LuaLatex doesn't need fontenc and uses UTF-8
% \usepackage{lmodern}  % Font face


% --------------------------------------------------------------------------------
% Page formatting
% --------------------------------------------------------------------------------
% Change the header/footer for chapter beginnings and normal pages
\usepackage[automark,headsepline]{scrlayer-scrpage}

% The package provides an easy and flexible user interface to customize the page
% layout, implementing auto-centering and auto-balancing mechanisms
% WARNING: WHEN CHANGING BCOR (Binding correction), the cover needs reworking!...
\newcommand{\theBCOR}{15mm}  % Define binding correction
\usepackage[
    bindingoffset=\theBCOR,
    % showframe, % Show boxes which indicate margins and paddings
    bottom = 3.5cm, % Margins
      left = 2.5cm,
     right = 2.5cm
] {geometry}

% The package 'float' provides a container for document objects which can not be
% broken over pages, such as tables and figures
% Needed for table and figure indexes  
\usepackage{float}

% support for landscape layout
\usepackage{lscape}

% support of \tablenotes command to add notes under table
\usepackage{threeparttable}

% To allow drawing more professional tables
\usepackage{booktabs}

% --------------------------------------------------------------------------------
% Contents
% --------------------------------------------------------------------------------
% Vector graphics (for Cover page)
\usepackage{tikz} 

% Allows additional parameters when including images
\usepackage{graphicx}

% Roman font family for all headings
\addtokomafont{disposition}{\rmfamily}

% Set the line spacing to 1.5
\usepackage[onehalfspacing]{setspace}

% Improves overall text spacing
% http://www.khirevich.com/latex/microtype/
\usepackage[stretch=10]{microtype}

% Math symbols like mu outside the math environment
\usepackage{textcomp}

% A comprehensive (SI) units package∗
% For defining SI units
\usepackage[
    range-units=single,         % Formatting ranges with single unit indication: 1 - 2 m
    range-phrase=-,             % Phrase for range: 1 - 2 m vs 1 to 2 m
    separate-uncertainty=true,  % sets +- between value and uncertainty 
    multi-part-units=repeat     % In expressions with multiple values (multi part numbers) 
                                % the unit is printed each time: 1 mm x 1 mm
] {siunitx}
% https://tex.stackexchange.com/questions/124488/multi-part-numbers-and-units-in-siunitx

% Allows Sourcecodes with highlighting 
\usepackage{listings}

% This package provides user control over the layout of the three basic list
% environments: enumerate, itemize and description
\usepackage{enumitem}
\setlist{nosep} % Remove the vertical space between \item elements in all lists

% ToDo Notes
% \setlength{\marginparwidth}{2cm}
\usepackage{todonotes}
\setuptodonotes{inline, inlinepar}
\reversemarginpar  % Put ToDo notes on the binding's side
% \usepackage{soul} % Colorful ToDo notes

% Check out colors here http://latexcolor.com/
\usepackage{xcolor}

\usepackage{amsmath}    % alignment of equations

% --------------------------------------------------------------------------------
% Other elements
% --------------------------------------------------------------------------------
% Blindtext: Organic looking text dummy
\usepackage{blindtext}

% Hyperlinks within the document (PDF)
% "hidelinks" hides visual highlighting of links
\usepackage[hidelinks]{hyperref}

% Package for Glossary and Index (Acronyms are listed in a separate list) 
\usepackage[acronym, nogroupskip]{glossaries}[=v4.49] % groupskip: alphabetic grouping of entries

\usepackage{xltabular}   % <------- FOR glossaries

% Integration and management of bibliographies
\usepackage{csquotes}   % backend=biber in biblatex needs this package
\usepackage[
    style=ieee,   % style of the bibliography, entries are sorted in alphabetic order. "ieee" is another common style.
    backend=biber,      % based on package 'biber' 
    bibencoding=ascii   % ASCII Text encoding; may use "utf8" instead
] {biblatex}

% --------------------------------------------------------------------------------
%                               PATHS & FILES
% --------------------------------------------------------------------------------
% Fix paths for standalone compiling
\ifstandalone
    \def \home {..}
\else
    \def \home {.}
\fi

% Package: scrlayer-scrpage
% \def \stylePath {\home/settings+/style/page}
\input{\home/settings+/style/page}  % Load page style

% Package: graphicx
\graphicspath{{\home/images/}}  % Set path to images

% Package: listings
\input{\home/settings+/style/code.tex}  % Set path to style file
\lstset{inputpath={\home/code/}} % Default path to code listings

% Package: glossaries
\input{\home/settings+/style/symbols}  % Set path to symbols list style file
\input{\home/settings+/style/acronyms}  % Set path to acronym list style file
% -------------------------------------------------------------------------------
%               Listing of all Glossary and Acronym Entries 
%                           use as shown below
% -------------------------------------------------------------------------------

% ==== EXEMPLARY ENTRY FOR SYMBOLS LIST =========================================

% ==== EXEMPLARY ENTRY FOR ACRONYMS LIST ========================================
% \newacronym{#label}{#acronym}{#long_form}

% define new command for custom arconym entry with only two arguments
% fabricates an easier way to use \newacronym 
\newcommand{\acroX}[2]{\newacronym{#1}{#1}{#2}}
% \acroX{label and arconym}{long name}
% \acroX{CD}               {Compact Disk}

\newcommand{\acroY}[3]{\newacronym{#1}{#2}{#3}}
% \arcoY{label}{acronym}{long name}
% \acroY{CD}   {cd}     {Compact Disk}
 
\newacronym{AEP}{AEP}{Imbalance price}
\newacronym{aFRR}{aFRR}{Automatic Frequency Restoration Reserve}


\newacronym{reBAP}{reBAP}{Uniform imbalance price}
\newacronym{TSO}{TSO}{Transmission System Operator}
\newacronym{FCR}{FCR}{Frequency Containment Reserve}
\newacronym{mFRR}{mFRR}{Manual Frequency Restoration Reserve}
\newacronym{BRP}{BRP}{Balancing Responsible Party}
\newacronym{SB}{SB}{System Balance}
\newacronym{VRE}{VRE}{variable renewable energy}
\newacronym{ID1}{ID1}{intraday index ID1}
\newacronym{MAE}{MAE}{mean average error}
\newacronym{RMSE}{RMSE}{root mean squared error}
\newacronym{MSE}{MSE}{mean squared error}
\newacronym{CRPS}{CRPS}{continuous ranked probabililty score}
\newacronym{GCC}{GCC}{Grid Control Cooperation}
\newacronym{IC}{IC}{Continuous intraday}
\newacronym{VWAP}{VWAP}{volume-weighted average price}
\newacronym{VID}{VID}{traded volume within the intraday market}
\newacronym{ID AEP}{ID AEP}{Intraday Average Energy Price}
\newacronym{FRR}{FRR}{Frequency Restoration Reserve}
\newacronym{TFT}{TFT}{Temporal Fusion Transformer}
\newacronym{DLM}{DLM}{Dynamic Linear Model}
\newacronym{GB}{GB}{Gradient Boosting}
\newacronym{RF}{RF}{Random Forest}
\newacronym{ARIMAX}{ARIMAX}{Autoregressive Integrated Moving Average with eXogenous variables}
\newacronym{xLSTM}{xLSTM}{Extended Long Short-Term Memory}
\newacronym{DWD}{DWD}{Deutscher Wetterdienst}
\newacronym{ENTSO-E}{ENTSO-E}{European Network of Transmission System Operators for Electricity}
\newacronym{IDA1}{IDA1}{Intraday auction 1}
\newacronym{MOSMIX}{MOSMIX}{Model Output Statistics-MIX}
\newacronym{mLSTM}{mLSTM}{memory-optimized LSTM}
\newacronym{sLSTM}{sLSTM}{speed-optimized LSTM}

% ==== EXEMPLARY ENTRY FOR MAIN GLOSSARY ========================================

    % \newglossaryentry{policy}{name={Policy},description={Im geschäftlichen Bereich bezeichnet Policy eine interne Leit- bzw. Richtlinie, die formal durch das Unternehmen dokumentiert und über ihr Management verantwortet wird}}
    % \newglossaryentry{pcie}{name={PCI Express},description={PCI Express („Peripheral Component Interconnect Express“, abgekürzt PCIe oder PCI-E) ist ein Standard zur Verbindung von Peripheriegeräten mit dem Chipsatz eines Hauptprozessors. PCIe ist der Nachfolger von PCI, PCI-X und AGP und bietet im Vergleich zu seinen Vorgängern eine höhere Datenübertragungsrate pro Pin.}}
    % \newglossaryentry{realnumber}
  % Load glossary, symbol and acronyms list

% Package: biblatex
\addbibresource{\home/references/references.bib}  % Set path to bib resources

% Custom variables
\input{\home/settings+/variables}
% --------------------------------------------------------------------------------
%                                   OPTIONAL
% --------------------------------------------------------------------------------


% Simple arithmetic for LaTeX commands
% \usepackage{calc}

% Document Elements
% -------------------

% Index
% \usepackage{imakeidx}

% compact Lists
%\usepackage{paralist}

% visual improvements for citations
% \usepackage{epigraph}

% Create pseudo code
% https://www.overleaf.com/learn/latex/Algorithms
% \usepackage{algorithm}
% \usepackage{algorithmic}
%\usepackage[noend]{algpseudocode}

% Formatting
% -------------------
% Tweaks for scrbook, redefines commands of other packages
% \usepackage{scrhack}

% Intelligent space separator (nice for superscript?)
% \usepackage{xspace}

% Allows breaks within tables
%\usepackage{tabularx}

% Allows for page breaks in tables
% \usepackage{longtable}

% allows modifying of captions
% \usepackage{caption}

% Multiline comments
%\usepackage{verbatim}

% % Custom colors
% \definecolor{dartmouthgreen}{rgb}{0.05, 0.5, 0.06}

% IF you want to define unicode characters
% \DeclareUnicodeCharacter{0229}{\c{e}}
% \DeclareUnicodeCharacter{0306}{\u{Z}}


% Document elements
% ------------------------------------

% Table package
% \usepackage{booktabs}

% Pie diagram
% \usepackage{datapie}

% Side by Side images
% \usepackage{subcaption}

% For landscape tables
%\usepackage{pdflscape}
%\usepackage{afterpage}

% Graphics can be flow around by text
%\usepackage{wrapfig}

\fi

% ----------------------------------------------------------------------------
%                                Introduction
% ----------------------------------------------------------------------------
\begin{document}

\chapter{Introduction}
\label{Chapter::Introduction}

In electrical power systems, the consumption and production of electricity must be balanced at all times for reasons of frequency stability\cite{weitemeyerIntegrationRenewableEnergy2015}. If the production outweighs the consumption there is a surplus of energy in the system and the frequency rises. Is the system short of energy due to more energy being drained from the grid than injected into it, the frequency drops.
In the German net, which this work will focus on, the frequency is set at 50hz. If the frequency deviates from this Transmission System Operators (TSOs) activate different balancing reserves to restore the net frequency. The balancing reserves consist of the frequency containment reserve (FCR), which is automatically used to keep the system in balance. This reserve is the fastest to act, providing energy in less than 30 seconds. This is followed by the automatic frequency restoration reserve is activated, which gradually replaces the FCR after 30 seconds, and finally the manual frequency restoration reserve, which can be activated to provide energy after 12.5 minutes. With these reserves TSOs ensure that the system remains in balance\cite{FrequencyReserves}.

%With more and more variable renewabl energy(VRE) sources being introduced, such as solar and wind parks, the inertia of the system decreases \cite{weitemeyerIntegrationRenewableEnergy2015}. 

In the German energy market all market players are called balancing responsible parties (BRP) \cite{narajewskiProbabilisticForecastingGerman2022}. 
They must trade excess generation or consumption with other BRPs in advance and submit the resulting schedules to the TSOs. 
These schedules may deviate from the physical volumes, resulting in an imbalance. All parties with imbalanced portfolios receive imbalance energy from the TSOs. 
The sum of all deviations in a balancing area is called the system balance (SB)\cite{eickeElectricityBalancingMarket2021}.

The TSOs calculate the price for the provided imbalance energy, this is called the imbalance price(AEP). For all control areas the uniform cross-control area imbalance price(reBAP) is used. 
Each BRP that contributed to the system imbalance, for example being short in energy while the system also is short in energy, has to pay the imbalance price for the volume needed to balance their portfolio. On the other hand BRPs that helped fight the system imbalance, for example having surplus energy while the system is short in energy, are compensated for stabilizing the net. They are paid the imbalance price for their imbalance volume\cite{eickeElectricityBalancingMarket2021}.

% With more and more variable renewabl energy(VRE) sources being introduced, such as solar and wind parks, the inertia of the system decreases \cite{weitemeyerIntegrationRenewableEnergy2015}. 


\section{Research Questions}
\label{Section::Research_Questions}
The goal of this thesis is to find out how well time series predictors work for the prediction of the imbalance price in the german energy market. 
For this task I will employ different models of variying complexity and compare their results.
With the informaiton produced by these models I am trying to get a better understanding of what information plays a large role for the reBAP.

Predicting the reBAP is a very challenging task, as many factors come into play.
Due to the reBAPs nature it is highly influenced by multiple factors, such as the system balance, current market situation and available balancing reserves. 
How these interact will be explained in section \ref{Section::Imbalance_Price}. 
The models used throughout this thesis will make their prediction about the reBAP one hour before energy delivery. 
%With this configuration there is still time for BRPs to trade energy to improve system stability. 


\section{Motivation}
\label{Section::Motivation}

With more and more variable renewable energy(VRE) sources being introduced into the net, such as solar and wind parks, the inertia of the system decreases \cite{weitemeyerIntegrationRenewableEnergy2015}. 
Generation of these can only be assumed using whether forecasts. If the actual wheathe deviates from the forecasts so does the energy production. In systems with a larger VRE share this can result in a larger system imbalance. 

To help stabilize the system a good forecast about the reBAP could be useful. A good estimate of the reBAP would give BRPs an idea whether action is required to help stabilize the energy net.

\section{Objectives}
\label{Section::Objectives}
This thesis will try to answer the question, whether time series predictors are able to make good predictions about future imbalance prices. To answer this question different state-of-the-art time series predictors will be compared on the task of predicting the imbalance price. 
Models that have previously been to tackle this task will be compared with new promising time series prediction models.

One hypothesis is, that new state-of-the-art time series predictors can outperform models previously employed on this problem, like the models presented in the next chapter. 

The results of the models are compared with the trades settled in the last hour before delivery, denoted as $ID_1$. Due to the imbalance price, BRPs are incentivised to trade electric power to balance their portfolio \cite{kochPASSIVEBALANCINGINTRADAY2020a}. The prices of trades close to the delivery time can be used as a proxy for the BRPs estimate of the imbalance price. Comparing the prediction accuracy against this index shows whether the models outperform the predictors currently used by BRPs.


 %In this research I will also look into whether these models work on live data. 
 The system imbalance is settled every 15 minutes, dividing a day into 96 timesteps \cite{NetztransparenzReBAP}.
 Each model will make a prediction for each timestep of a day, using the data available at prediction time.
 % Part of the data used for the prediction consists of live data, published for each time step. 
%The latest possible time of prediction is five minutes before the delivery period \cite{EPEXTradingBrochure}.
 %Some of the live data for timestep $t-1$ is delayed, making it unavailable for a prediction for $t+1$ at the prediction timestep $t$. 
 %This thesis will try to find the best point in time to make a prediction about the imbalance of a timestep in the future.

%A second hypothesis is that a prediction, which utilizes the latest data is more accurate, than a prediction made at an earlier point in time.


\section{Methodology}
\label{Section::Methodology}
This thesis will try to answer the question of how well time series forecasting models are able to predict prices in the German electricity market. The performance of the models will be measured by their accuracy in predicting the imbalance price. A benchmark for this prediction will be the $ID_1$ price. Since the $ID_1$ price is the average of the trades in the hour before delivery, it serves as a proxy for the BRPs prediction of the imbalance price \cite{narajewskiProbabilisticForecastingGerman2022}. 

In this thesis I will use benchmarks, such as the previous state of the art methods for imbalance price prediction. These benchmarks will be compared with other novel time series predictors such as xLSTM, iTransformers, State Space Models and Recurrent Neural Networks to see how they perform in this domain.

The data used in these predictions will come from different sources: First, data from $netztransparenz.de$ will be used. This contains information about the past time intervals, consisting of the system imbalance, an estimate of the imbalance price and the GCC traffic light which shows the current system balance status \cite{NetztransparenzTrafficLight}. Secondly, weather data provided by the DWD is used. This data consists of information about global radiation, wind speeds and directions, and various other measures that have been condensed into a data set called $dwd-mosmix$. A more detailed overview of the data can be found here: \cite{DWDMosmix}. The data from DWD is obtained from a range of weather stations across Germany. The third type of data used in this thesis is financial data provided by the european power exchange (EPEX). This data consists of prices of different energy-auctions, as well as information on recent trades.
The European Network of Transmission System Operators for Electricity (ENTSO-E) publishes a wide range of data for the european power network. From this datasource data about forecasted load, forecasted generation will be used in this thesis. 

All of these models will be implemented using the Google Cloud environment. The data will be stored in Google's BigQuery and the models will be implemented in Vertex AI, using a virtual machine.

\section{Structure}
\label{Section::Structure}
The following chapter \ref{Chapter::Theory} contains information about how the german electricity market works and how the imbalance price is calculated
In chapter \ref{Chapter::Sota} I will show some relevant research done on this problem domain, including other models that have already been tested on the prediction of the german imbalance price. 
That chapter will also contain information about the state of the art time series predictors used in this thesis.

Information about the methodology used can be found in chapter \ref{Chapter::Implementation}, where I explain how I received and handled the data used for the training, as well as how my experiments are set up.

That chapter is followed by chapter \ref{Chapter::Implementation} in which I go over model design and archtitecutreq ??

After that chapter \ref{Chapter::Experiments} contains the different experiments, followed by the Chapter \ref{Chapter:Results} which contains the results to those experiments.

This thesis concludes with the chapter \ref{Chapter::Discussion} where the results are put into perspective and \ref{Chapter::Conclusion} where all findings are summarized.

\ifstandalone
    \printglossary
    \printbibliography[heading=bibintoc]
\fi

\end{document}

\documentclass[class=scrbook, crop=false]{standalone}
\usepackage[subpreambles=true]{standalone}
\ifstandalone
    \input{../settings+/settings}
\fi
\usepackage[table]{xcolor}% http://ctan.org/pkg/xcolor

% ----------------------------------------------------------------------------
%                               Implementation
% ----------------------------------------------------------------------------
\begin{document}

\chapter{Implementation} % Outline text
\label{Chapter::Implementation}
    This chapter contains a description of how the implemented algorithm works.

% First explain concepts you used in your thesis like filters or methods
% Then explain your approach or algorithm
% Use flowcharts to give an overview
\section{Feature engineering}
\label{Section::Feature_engineering}

The data introduced in section \ref{Chapter::Methodology} can be further refined to gain additional information. 
The following sections contain information about how the data was used to create more features.
Feature engineering was done for energy data, weather data and holiday data.

    \subsection{Energy data}
    \label{Section::Energy_Data}
    The energy data provided by ENTSO-E can be further refined to gain additional information.
    One additional feature is the residual load. 
    Residual load is defined as the load that is not covered by VRE. 
    This is calulated by subtracting the generation provided by solar, wind onshore and win offshore from the load.
    
    Another way of generating more informative features is to calculate the forecasting error. 
    For this the forecast for solar generation, wind generation and load is substracted from the actual measured values.
    This can also be done for the residual load.
    For each of these variables the correlation with the imbalance price is calclulated to estimate how much influence the variables have in determining the imbalance price.
    Table \ref{Table::Rebap_Correlations_ENTSOE} contains the results for these calculations. 
    
    Out of these features the forecasted and actual value for residual load, the forecasting error for solar generation have the largest absolute correlation.
    It should be noted that the actual values are not available at prediction time.    

    \begin{table}
    \begin{tabular}{l|l|l|l}
    Variable Name &Forecasted Value& Actual Value& Forecasting Error \\\hline
    Solar &-0,053& -0,064& \cellcolor{green}  -0,116 \\
    Wind offshore & -0,077&-0,075& \cellcolor{green} 0,000 \\
    Wind onshore &-0,101&-0,107& \cellcolor{green} -0,047 \\
    Load &0,070&0,056& \cellcolor{green} -0,037 \\
    Residual Load & \cellcolor{green} 0,169& \cellcolor{green}0,171& \cellcolor{green}0,025\\
    \end{tabular}
    
    \caption{Correlations between variables and imbalance price (rounded to next $10^{-3}$). Green cells are new features}
    \label{Table::Rebap_Correlations_ENTSOE}
    \end{table}

\end{document}

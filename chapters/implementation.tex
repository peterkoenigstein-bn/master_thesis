\documentclass[class=scrbook, crop=false]{standalone}
\usepackage[subpreambles=true]{standalone}
\ifstandalone
    \input{../settings+/settings}
\fi
\usepackage[table]{xcolor}% http://ctan.org/pkg/xcolor

% ----------------------------------------------------------------------------
%                               Implementation
% ----------------------------------------------------------------------------
\begin{document}

\chapter{Implementation} % Outline text
\label{Chapter::Implementation}
    This chapter contains a description of how the implemented algorithm works.

% First explain concepts you used in your thesis like filters or methods
% Then explain your approach or algorithm
% Use flowcharts to give an overview
\section{Feature engineering}
\label{Section::Feature_engineering}

The data introduced in section \ref{Chapter::Methodology} can be further refined to gain additional information. 
The following sections contain information about how the data was used to create more features.
Feature engineering was done for energy data, weather data and holiday data.

In the following sections the correlation between features and the imbalance price will be calculated.
The distribution of the imbalance price has been inspected. 
To determine which correlation method to use the distribution of the imbalance price has been checked.
Furthermore the skewness and kurtosis of the imbalance price has been calculated.

\begin{figure}[ht]
            \centering
            \includegraphics[width=.5\textwidth]{implementation/probability_plot.png}
            \caption[Probability plot for the imbalance price]{Probability plot for the imbalance price}
            \label{fig::rebap_distribution}
 \end{figure}

Using the anderson darling test \ref{andersonAsymptoticTheoryCertain1952} it was confirmed, that the data is normal distributed.
A probability plot for the imbalance price can be seen in figure \ref{fig::rebap_distribution}. 
The red line resembles a normal distribution. 
With a skewness of $27.08$ and a kurtosis of $1194.02$ the pearson correlation can not be used.
Instead for the following correlation analyses the spearman rank correlation will be used.

The imbalance volume and the imbalance price are highly correlated. 
Figure \ref{Figure::volume_price_scatter} shows scatter plots for the imbalance volume and imbalance price.
The correlation coefficient between the imbalance price and the imbalance volume is $0.774$.
In the following subsections the correlation between newly engineered variables and the imbalance price will be inspected.
Due to the high correlation with the imbalance volume, correlations with the imbalance volumes will also be regarded.

\begin{figure}[ht]
            \centering
            \includegraphics[width=.45\textwidth]{implementation/nrv_rebap_scatter.png}
            \includegraphics[width=.45\textwidth]{implementation/nrv_rebap_scatter_clipped.png}
            \caption[Scatter plots for imbalance price and imbalance volume. For the plot on the right the imbalane price has been clipped to give information about non-outliers]{Scatter plots for imbalance price and imbalance volume. For the plot on the right the imbalane price has been clipped to give information about non-outliers}
            \label{Figure::volume_price_scatter}
\end{figure}

    \subsection{Energy data}
    \label{Section::FE_Energy_Data}
    The energy data provided by ENTSO-E can be further refined to gain additional information.
    One additional feature is the residual load. 
    Residual load is defined as the load that is not covered by VRE. 
    This is calulated by subtracting the generation provided by solar, wind onshore and win offshore from the load.
    
    Another way of generating more informative features is to calculate the forecasting error. 
    For this the forecast for solar generation, wind generation and load is substracted from the actual measured values.
    This can also be done for the residual load.
    For each of these variables the correlation with the imbalance price is calclulated to estimate how much influence the variables have in determining the imbalance price.
    Table \ref{Table::Rebap_Correlations_ENTSOE} contains the results for these calculations. 
    
    Out of these features the forecasted and actual value for residual load, the forecasting error for solar generation have the largest absolute correlation.
    It should be noted that the actual values are not available at prediction time.    

    \begin{table}
    \centering
    \begin{tabular}{l|l|l|l}
    Variable Name	&Forecasted Value& Actual Value	& Forecasting Error \\\hline
    Solar 		& -0.083		& -0.105		& \cellcolor{green} -0.155 \\
    Wind offshore 	& -0.164		& -0.152		& \cellcolor{green} 0.008 \\
    Wind onshore 	& -0.176		& \textbf{-0.189}	& \cellcolor{green} -0.104 \\
    Load 		&0.101		& 0.121		& \cellcolor{green}  0.132 \\
    Residual Load 	& \cellcolor{green} \textbf{0.343}& \cellcolor{green} \textbf{0.391}& \cellcolor{green}0.192\\
    \end{tabular}

    
    \caption{Correlations between variables and imbalance price (rounded to next $10^{-3}$). Green cells are new features}
    \label{Table::Rebap_Correlations_ENTSOE}
    \end{table}

    \begin{table}
    \centering
    \begin{tabular}{l|l|l|l}
    Variable Name	&Forecasted Value& Actual Value	& Forecasting Error \\\hline
    Solar 		&  -0.011		& -0.024		& \cellcolor{green} \textbf{-0.149} \\
    Wind offshore 	& -0.013		&  -0.022		& \cellcolor{green}\textbf{0.114}\\
    Wind onshore 	& 0.010		& 0.008		& \cellcolor{green} -0.093\\
    Load 		& 0.041		& 0.066		& \cellcolor{green}  -0.040 \\
    Residual Load 	& \cellcolor{green} 0.046& \cellcolor{green} 0.080& \cellcolor{green} \textbf{0.193}\\
    \end{tabular}

    
    \caption{Correlations between variables and imbalance price (rounded to next $10^{-3}$). Green cells are new features}
    \label{Table::Imbalance_volume_Correlations_ENTSOE}
    \end{table}
    
    The newly introduced features in this section will also be used for inspecting possible features introduced at a later point.
    
    \subsection{Weather data}
    \label{Section::Weather_data}
    The weather data contains a lot of datapoints, especially the MOSMIX forecast. 
    For each of the originally 40 features a forecast is done 240 times for each timestep. 
    This data needs to be condensed, as not all of theses timestamps are useful.
    There are certain timestamps which will be inspected closer for the feature engineering.
    
    The forecasts done by ENTSO-E are published the day before and use the weather forecast from that time.
    The forecast happens at 18:00. 
    In the MOSMIX data this timestamp will be more closely looked at for this reason.
    Another important timestamp is the latest timestamp. 
    With a forecast being done each hour, the latest available timestamp at prediction time is the one 2 hours before gate closure time, due to the prediction happening at 1 hour before gate closure time.
    For each of the variables introduced in \ref{Table::DWD_MOSMIX_Parameters_Small} the correlation for both of the previously discussed timestamps is checked.
    
    With both of these configurations for each of the variables a new feature can be created. 
    To get an estimate for the forecasting error in the ENTSO-E data, the change in forecast can be calculated in the DWD data.
    Under the assumption, that a more recent forecast is more accurate, the difference between the latest forecast and the forecast of the day before at $18:00$ is calculated.
    The names for these new features are created by combining a prefix with the abbreviation for the MOSMIX feature.
    For the forecast from the previous day the prefix \textit{DayAhead} is used, the prefix \textit{LastFC} is used for the most recent forecast and \textit{Diff} denotes that the features is the difference between the day ahead forecast and the most recent forecast.
    The results for the correlation can be found in table \ref{Table::DWD_MOSMIX_correlations}. 
    
    \begin{table}
    \centering
    \begin{tabular}{l|l}
    Variable Name	& Correlation coefficient \\\hline
   	LastFC\_FX1	&-0.254\\
	DayAhead\_FX1	&-0.25\\
	LastFC\_FF		&-0.247\\
	DayAhead\_FF 	&-0.243\\
	DayAhead\_N05	&0.185\\
	LastFC\_N05	&0.176\\
	DayAhead\_wwM	&0.167\\
	LastFC\_wwM	&0.162\\
	LastFC\_Rad1h 	&-0.128\\
	Diff\_Rad1h          	&-0.115\\
	LastFC\_SunD1     	&-0.102\\
    \end{tabular}
    
    \caption{Correlations between feature engineered MOSMIX variables and imbalance price with an absolute correlation coefficient larger than $0.1$ (rounded to next $10^{-3}$)}
    \label{Table::imbalance_price_MOSMIX_correlations}
    \end{table}
    
    \begin{table}
    \centering
    \begin{tabular}{l|l}
    Variable Name	& Correlation coefficient \\\hline
	Diff\_Rad1h                          &-0.08\\
	LastFC\_N05                          &0.074\\
	DayAhead\_N05                        &0.073\\
	Diff\_TTT                           &-0.068\\
	Diff\_SunD1                         &-0.061\\
	Diff\_T5cm                          &-0.055\\
	Diff\_Nl                             &0.052\\
    \end{tabular}
    
    \caption{Correlations between feature engineered MOSMIX variables and imbalance volume (rounded to next $10^{-3}$)}
    \label{Table::imbalance_volume_MOSMIX_correlations}
    \end{table}
    
    
    \begin{table}
    \centering
    \begin{tabular}{l|l}
    Variable Name	& Correlation coefficient \\\hline
	LastFC\_T5cm                        &-0.258\\
	DayAhead\_T5cm                      &-0.253\\
	LastFC\_TTT                         &-0.249\\
	DayAhead\_FF                         &0.246\\
	DayAhead\_TTT                       &-0.243\\
	LastFC\_FF                           &0.233\\
	LastFC\_Nl                           &0.221\\
	DayAhead\_FX1                        &0.216\\
	DayAhead\_Nl                         &0.205\\
	LastFC\_FX1                          &0.204\\
    \end{tabular}
    
    \caption{Correlations between feature engineered MOSMIX variables and residual load forecasting error with an absolute correlation coefficient larger than $0.2$ (rounded to next $10^{-3}$)}
    \label{Table::residual_load_fce_MOSMIX_correlations}
    \end{table}

    Dividing the most recent forecast by the day ahead forecast would maybe provide a more meaningful relation, but due to the nature of the data this is problematic.
    The dataset contains variables with large value ranges, including 0. 
    By dividing, this new feature might take extreme values.
    On the other hand some variables have scales which make the division less meaningful.
    For example the temperature is measured in kelvin, leading to small quotients.

\section{Data Split}
\label{Section::Data_Split}

To make sure the models trained during the experiments are comparable the models have to be trained and evaulated on the same dataset.
The data was split into a set of training, validation and test data with a share of 70\%, 10\% and 20\% respectively.

For random forest and arima this data split could be done using $train\_test\_split$ from the sklearn package. 
With the more complex Time Series predictors xLSTM and iTransformer needing a different input shape this is a bit more complex.

Instead the training, validation and test split was done along the temporal axis. 
The oldest 70\% were defined as training data, the newest 20\% as test data, with the remaining 10\% being the validation data.


\section{Models}
\label{Section::Models}

In this section implementation details for each of the used models will be discussed. 
The hyperparameter configurations might change during the experiments. 
Any changes done specifically for an experiment will be discussed in the experiment's section.

All the models were implemented in python. 

\subsection{Random Forest}
\label{Section::Random_Forest}

As a random forest regressor the RandomForestRegressor provided by sklearn was used.
Random forest can not use time series data out of the box.
To give some more information about past timesteps shifted features are used.

For the training in the experiments the timesteps of the previous 4 quarter hours were used.
Also included were the same timestamps for the previous 2 weeks.

To find the best random forest regressor RandomizedSearchCV by sklearn was used.
In total 20 different parameter combination were tested with a cross validation of 5.
The performance of the models was measured using the negative mean squared error.

\subsection{Arima}
\label{Section::Arima}

For the training of the arima model the module pmdarima was used.
This module provides the function \textit{auto\_arima} to find the best arima model.
The model was trained using the target variable as well as exogenous variables.
The training time increases with the amount of exogenous variables.
For that reason only a subset of features was used for this training.

After the training of a random forest the importance of its features can be extracted using \textit{feature\_importances}. 
The $10$ most important features were used for the training of the arima model.
Using \textit{auto\_arima} the best parameter configuration can be found.

\subsection{xLSTM}
\label{Section::xLSTM}

\subsection{iTransformer}
\label{Section::iTransformer}



\end{document}

\documentclass[class=scrbook, crop=false]{standalone}
\usepackage[subpreambles=true]{standalone}
\ifstandalone
    % WARNING: Proceed with caution!

% -----------------------------------------------------------------------------------
% For package standalone
% -----------------------------------------------------------------------------------
\usepackage{import}

% -----------------------------------------------------------------------------------
% Language and typeset
% -----------------------------------------------------------------------------------
\usepackage[ngerman, english]{babel}

\usepackage{subcaption}
% Umlauts and other special characters (UTF-8)
% \usepackage[utf8]{inputenc}
\usepackage{fontspec}
\setsansfont{Arial}
% \usepackage[T1]{fontenc}  % Enable accented characters and umlauts
% LuaLatex doesn't need fontenc and uses UTF-8
% \usepackage{lmodern}  % Font face


% --------------------------------------------------------------------------------
% Page formatting
% --------------------------------------------------------------------------------
% Change the header/footer for chapter beginnings and normal pages
\usepackage[automark,headsepline]{scrlayer-scrpage}

% The package provides an easy and flexible user interface to customize the page
% layout, implementing auto-centering and auto-balancing mechanisms
% WARNING: WHEN CHANGING BCOR (Binding correction), the cover needs reworking!...
\newcommand{\theBCOR}{15mm}  % Define binding correction
\usepackage[
    bindingoffset=\theBCOR,
    % showframe, % Show boxes which indicate margins and paddings
    bottom = 3.5cm, % Margins
      left = 2.5cm,
     right = 2.5cm
] {geometry}

% The package 'float' provides a container for document objects which can not be
% broken over pages, such as tables and figures
% Needed for table and figure indexes  
\usepackage{float}

% support for landscape layout
\usepackage{lscape}

% support of \tablenotes command to add notes under table
\usepackage{threeparttable}

% To allow drawing more professional tables
\usepackage{booktabs}

% --------------------------------------------------------------------------------
% Contents
% --------------------------------------------------------------------------------
% Vector graphics (for Cover page)
\usepackage{tikz} 

% Allows additional parameters when including images
\usepackage{graphicx}

% Roman font family for all headings
\addtokomafont{disposition}{\rmfamily}

% Set the line spacing to 1.5
\usepackage[onehalfspacing]{setspace}

% Improves overall text spacing
% http://www.khirevich.com/latex/microtype/
\usepackage[stretch=10]{microtype}

% Math symbols like mu outside the math environment
\usepackage{textcomp}

% A comprehensive (SI) units package∗
% For defining SI units
\usepackage[
    range-units=single,         % Formatting ranges with single unit indication: 1 - 2 m
    range-phrase=-,             % Phrase for range: 1 - 2 m vs 1 to 2 m
    separate-uncertainty=true,  % sets +- between value and uncertainty 
    multi-part-units=repeat     % In expressions with multiple values (multi part numbers) 
                                % the unit is printed each time: 1 mm x 1 mm
] {siunitx}
% https://tex.stackexchange.com/questions/124488/multi-part-numbers-and-units-in-siunitx

% Allows Sourcecodes with highlighting 
\usepackage{listings}

% This package provides user control over the layout of the three basic list
% environments: enumerate, itemize and description
\usepackage{enumitem}
\setlist{nosep} % Remove the vertical space between \item elements in all lists

% ToDo Notes
% \setlength{\marginparwidth}{2cm}
\usepackage{todonotes}
\setuptodonotes{inline, inlinepar}
\reversemarginpar  % Put ToDo notes on the binding's side
% \usepackage{soul} % Colorful ToDo notes

% Check out colors here http://latexcolor.com/
\usepackage{xcolor}

\usepackage{amsmath}    % alignment of equations

% --------------------------------------------------------------------------------
% Other elements
% --------------------------------------------------------------------------------
% Blindtext: Organic looking text dummy
\usepackage{blindtext}

% Hyperlinks within the document (PDF)
% "hidelinks" hides visual highlighting of links
\usepackage[hidelinks]{hyperref}

% Package for Glossary and Index (Acronyms are listed in a separate list) 
\usepackage[acronym, nogroupskip]{glossaries}[=v4.49] % groupskip: alphabetic grouping of entries

\usepackage{xltabular}   % <------- FOR glossaries

% Integration and management of bibliographies
\usepackage{csquotes}   % backend=biber in biblatex needs this package
\usepackage[
    style=ieee,   % style of the bibliography, entries are sorted in alphabetic order. "ieee" is another common style.
    backend=biber,      % based on package 'biber' 
    bibencoding=ascii   % ASCII Text encoding; may use "utf8" instead
] {biblatex}

% --------------------------------------------------------------------------------
%                               PATHS & FILES
% --------------------------------------------------------------------------------
% Fix paths for standalone compiling
\ifstandalone
    \def \home {..}
\else
    \def \home {.}
\fi

% Package: scrlayer-scrpage
% \def \stylePath {\home/settings+/style/page}
\input{\home/settings+/style/page}  % Load page style

% Package: graphicx
\graphicspath{{\home/images/}}  % Set path to images

% Package: listings
\input{\home/settings+/style/code.tex}  % Set path to style file
\lstset{inputpath={\home/code/}} % Default path to code listings

% Package: glossaries
\input{\home/settings+/style/symbols}  % Set path to symbols list style file
\input{\home/settings+/style/acronyms}  % Set path to acronym list style file
% -------------------------------------------------------------------------------
%               Listing of all Glossary and Acronym Entries 
%                           use as shown below
% -------------------------------------------------------------------------------

% ==== EXEMPLARY ENTRY FOR SYMBOLS LIST =========================================

% ==== EXEMPLARY ENTRY FOR ACRONYMS LIST ========================================
% \newacronym{#label}{#acronym}{#long_form}

% define new command for custom arconym entry with only two arguments
% fabricates an easier way to use \newacronym 
\newcommand{\acroX}[2]{\newacronym{#1}{#1}{#2}}
% \acroX{label and arconym}{long name}
% \acroX{CD}               {Compact Disk}

\newcommand{\acroY}[3]{\newacronym{#1}{#2}{#3}}
% \arcoY{label}{acronym}{long name}
% \acroY{CD}   {cd}     {Compact Disk}
 
\newacronym{AEP}{AEP}{Imbalance price}
\newacronym{aFRR}{aFRR}{Automatic Frequency Restoration Reserve}


\newacronym{reBAP}{reBAP}{Uniform imbalance price}
\newacronym{TSO}{TSO}{Transmission System Operator}
\newacronym{FCR}{FCR}{Frequency Containment Reserve}
\newacronym{mFRR}{mFRR}{Manual Frequency Restoration Reserve}
\newacronym{BRP}{BRP}{Balancing Responsible Party}
\newacronym{SB}{SB}{System Balance}
\newacronym{VRE}{VRE}{variable renewable energy}
\newacronym{ID1}{ID1}{intraday index ID1}
\newacronym{MAE}{MAE}{mean average error}
\newacronym{RMSE}{RMSE}{root mean squared error}
\newacronym{MSE}{MSE}{mean squared error}
\newacronym{CRPS}{CRPS}{continuous ranked probabililty score}
\newacronym{GCC}{GCC}{Grid Control Cooperation}
\newacronym{IC}{IC}{Continuous intraday}
\newacronym{VWAP}{VWAP}{volume-weighted average price}
\newacronym{VID}{VID}{traded volume within the intraday market}
\newacronym{ID AEP}{ID AEP}{Intraday Average Energy Price}
\newacronym{FRR}{FRR}{Frequency Restoration Reserve}
\newacronym{TFT}{TFT}{Temporal Fusion Transformer}
\newacronym{DLM}{DLM}{Dynamic Linear Model}
\newacronym{GB}{GB}{Gradient Boosting}
\newacronym{RF}{RF}{Random Forest}
\newacronym{ARIMAX}{ARIMAX}{Autoregressive Integrated Moving Average with eXogenous variables}
\newacronym{xLSTM}{xLSTM}{Extended Long Short-Term Memory}
\newacronym{DWD}{DWD}{Deutscher Wetterdienst}
\newacronym{ENTSO-E}{ENTSO-E}{European Network of Transmission System Operators for Electricity}
\newacronym{IDA1}{IDA1}{Intraday auction 1}
\newacronym{MOSMIX}{MOSMIX}{Model Output Statistics-MIX}
\newacronym{mLSTM}{mLSTM}{memory-optimized LSTM}
\newacronym{sLSTM}{sLSTM}{speed-optimized LSTM}

% ==== EXEMPLARY ENTRY FOR MAIN GLOSSARY ========================================

    % \newglossaryentry{policy}{name={Policy},description={Im geschäftlichen Bereich bezeichnet Policy eine interne Leit- bzw. Richtlinie, die formal durch das Unternehmen dokumentiert und über ihr Management verantwortet wird}}
    % \newglossaryentry{pcie}{name={PCI Express},description={PCI Express („Peripheral Component Interconnect Express“, abgekürzt PCIe oder PCI-E) ist ein Standard zur Verbindung von Peripheriegeräten mit dem Chipsatz eines Hauptprozessors. PCIe ist der Nachfolger von PCI, PCI-X und AGP und bietet im Vergleich zu seinen Vorgängern eine höhere Datenübertragungsrate pro Pin.}}
    % \newglossaryentry{realnumber}
  % Load glossary, symbol and acronyms list

% Package: biblatex
\addbibresource{\home/references/references.bib}  % Set path to bib resources

% Custom variables
\input{\home/settings+/variables}
% --------------------------------------------------------------------------------
%                                   OPTIONAL
% --------------------------------------------------------------------------------


% Simple arithmetic for LaTeX commands
% \usepackage{calc}

% Document Elements
% -------------------

% Index
% \usepackage{imakeidx}

% compact Lists
%\usepackage{paralist}

% visual improvements for citations
% \usepackage{epigraph}

% Create pseudo code
% https://www.overleaf.com/learn/latex/Algorithms
% \usepackage{algorithm}
% \usepackage{algorithmic}
%\usepackage[noend]{algpseudocode}

% Formatting
% -------------------
% Tweaks for scrbook, redefines commands of other packages
% \usepackage{scrhack}

% Intelligent space separator (nice for superscript?)
% \usepackage{xspace}

% Allows breaks within tables
%\usepackage{tabularx}

% Allows for page breaks in tables
% \usepackage{longtable}

% allows modifying of captions
% \usepackage{caption}

% Multiline comments
%\usepackage{verbatim}

% % Custom colors
% \definecolor{dartmouthgreen}{rgb}{0.05, 0.5, 0.06}

% IF you want to define unicode characters
% \DeclareUnicodeCharacter{0229}{\c{e}}
% \DeclareUnicodeCharacter{0306}{\u{Z}}


% Document elements
% ------------------------------------

% Table package
% \usepackage{booktabs}

% Pie diagram
% \usepackage{datapie}

% Side by Side images
% \usepackage{subcaption}

% For landscape tables
%\usepackage{pdflscape}
%\usepackage{afterpage}

% Graphics can be flow around by text
%\usepackage{wrapfig}

\fi
\usepackage[table]{xcolor}% http://ctan.org/pkg/xcolor

% ----------------------------------------------------------------------------
%                               Implementation
% ----------------------------------------------------------------------------
\begin{document}

\chapter{Implementation} % Outline text
\label{Chapter::Implementation}
    This chapter contains a description of how the implemented algorithm works.

% First explain concepts you used in your thesis like filters or methods
% Then explain your approach or algorithm
% Use flowcharts to give an overview
\section{Feature engineering}
\label{Section::Feature_engineering}

The data introduced in section \ref{Chapter::Methodology} can be further refined to gain additional information. 
The following sections contain information about how the data was used to create more features.
Feature engineering was done for energy data, weather data and holiday data.

    \subsection{Energy data}
    \label{Section::Energy_Data}
    The energy data provided by ENTSO-E can be further refined to gain additional information.
    One additional feature is the residual load. 
    Residual load is defined as the load that is not covered by VRE. 
    This is calulated by subtracting the generation provided by solar, wind onshore and win offshore from the load.
    
    Another way of generating more informative features is to calculate the forecasting error. 
    For this the forecast for solar generation, wind generation and load is substracted from the actual measured values.
    This can also be done for the residual load.
    For each of these variables the correlation with the imbalance price is calclulated to estimate how much influence the variables have in determining the imbalance price.
    Table \ref{Table::Rebap_Correlations_ENTSOE} contains the results for these calculations. 
    
    Out of these features the forecasted and actual value for residual load, the forecasting error for solar generation have the largest absolute correlation.
    It should be noted that the actual values are not available at prediction time.    

    \begin{table}
    \begin{tabular}{l|l|l|l}
    Variable Name &Forecasted Value& Actual Value& Forecasting Error \\\hline
    Solar &-0,053& -0,064& \cellcolor{green} \textbf{-0,116} \\
    Wind offshore & -0,077&-0,075& \cellcolor{green} 0,000 \\
    Wind onshore &-0,101&-0,107& \cellcolor{green} -0,047 \\
    Load &0,070&0,056& \cellcolor{green} -0,037 \\
    Residual Load & \cellcolor{green} \textbf{0,169}& \cellcolor{green} \textbf{0,171}& \cellcolor{green}0,025\\
    \end{tabular}
    
    \caption{Correlations between variables and imbalance price (rounded to next $10^{-3}$). Green cells are new features}
    \label{Table::Rebap_Correlations_ENTSOE}
    \end{table}
    
    The newly introduced features in this section will also be used for inspecting possible features introduced at a later point.
    
    \subsection{Weather data}
    \label{Section::Weather_data}
    The weather data contains a lot of datapoints, especially the MOSMIX forecast. 
    For each of the originally 40 features a forecast is done 240 times for each timestep. 
    This data needs to be condensed, as not all of theses timestamps are useful.
    There are certain timestamps which will be inspected closer for the feature engineering.
    
    The forecasts done by ENTSO-E are published the day before and use the weather forecast from that time.
    The forecast happens at 18:00. 
    In the MOSMIX data this timestamp will be more closely looked at for this reason.
    Another important timestamp is the latest timestamp. 
    With a forecast being done each hour, the latest available timestamp at prediction time is the one 2 hours before gate closure time, due to the prediction happening at 1 hour before gate closure time.
    For each of the variables introduced in \ref{Table::DWD_MOSMIX_Parameters_Small} the correlation for both of the previously discussed timestamps is checked.
    The results for the correlation can be found in table \ref{Table::DWD_MOSMIX_correlations}. 
    
    With both of these configurations for each of the variables a new feature can be created. 
    To get an estimate for the forecasting error in the ENTSO-E data, the change in forecast can be calculated in the DWD data.
    Under the assumption, that a more recent forecast is more accurate, the difference between the latest forecast and the forecast of the day before at $18:00$ is calculated. 
    
    Dividing the most recent forecast by the day ahead forecast would maybe provide a more meaningful relation, but due to the nature of the data this is problematic.
    The dataset contains variables with large value ranges, including 0. 
    By dividing, this new feature might take extreme values.
    On the other hand some variables have scales which make the division less meaningful.
    For example the temperature is measured in kelvin, leading to small quotients.

    

\end{document}
